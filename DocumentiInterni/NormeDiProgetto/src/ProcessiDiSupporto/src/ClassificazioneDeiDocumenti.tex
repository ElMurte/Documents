\subsection{Classificazione dei documenti}
Tutti i documenti prodotti dal gruppo {\Gruppo} necessitano dell’approvazione del Responsabile di progetto prima della loro divulgazione.

\subsubsection{Documentazione Interna}
La documentazione interna è un particolare tipo di documentazione che verrà divulgata solamente internamente al gruppo {\Gruppo}. Fanno parte della documentazione interna i seguenti tipi di documenti:
\begin{itemize}
	\item verbali Interni;
	\item studio di fattibilità;
	\item norme di progetto.
\end{itemize}

\subsubsection{Documentazione Esterna}
La documentazione esterna è un tipo di documentazione cui divulgazione sarà esterna al gruppo {\Gruppo}. Fanno parte della documentazione esterna i seguenti tipi di documenti:
\begin{itemize}
	\item analisi dei requisiti;
	\item piano di qualifica;
	\item piano di progetto;
	\item lettera di presentazione;
	\item glossario.
\end{itemize}

\subsubsection{Stesura dei verbali}
Al momento della redazione del verbale, un membro del gruppo {\Gruppo} sarà scelto, a rotazione, come Redattore dello stesso. Per ogni incontro del gruppo verrà prodotto un verbale ad uso, come specificato sopra, solamente interno. Una volta steso lo stesso ed approvato, non è prevista la possibilità di modifica dello stesso, in quanto richiederebbe una modifica di azioni e/o tecnologie già discusse in sedute precedenti in modo retroattivo. La struttura del verbale è qui descritta:
\begin{itemize}
	\item luogo svolgimento riunione;
	\item data svolgimento riunione;
	\item ora di inizio;
	\item ora di fine;
	\item membri del gruppo presenti con eventuali esterni.
\end{itemize}

A questo segue l’Ordine del giorno, ovvero una descrizione sommaria degli argomenti da discutere nel corso del meeting. Successivamente, è trascritto il verbale della riunione, ovvero una descrizione più dettagliata di tutti gli argomenti discussi e le decisioni prese durante il corso del meeting. Ogni verbale ha la seguente nomenclatura:
\begin{center}
\textbf{Verbale\_I/E\_ YYYY-MM-DD}\\
\end{center}
dove \textbf{I/E} rappresenta il tipo del verbale secondo la seguente sintassi:
\begin{itemize}
	\item I: se il verbale è ad uso interno al gruppo;
	\item E: se il verbale è ad uso interno al gruppo, con la partecipazione di individui esterni allo stesso;
\end{itemize}
La nomenclatura del verbale si attiene allo standard ISO 8601, con le date scritte nel seguente formato:
\begin{itemize}
	\item YYYY: rappresenta l’anno con 4 cifre;
	\item MM: rappresenta il mese con 2 cifre;
	\item DD: rappresenta il giorno con 2 cifre.
\end{itemize}

Eventuali orari presenti, verranno trascritti secondo la notazione ISO 8601, che prevede l’uso dei numeri in un intervallo da 0 a 24 per rappresentare l’ora, e da 0 a 59 per rappresentare i minuti e secondi.
