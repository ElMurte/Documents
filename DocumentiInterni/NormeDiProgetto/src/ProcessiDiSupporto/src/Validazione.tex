\subsection{Validazione}
Lo scopo della validazione è accertare che il prodotto finale soddisfi il compito per il quale è stato creato, ovvero che rispetti i vincoli obbligatori del committente. Questo processo quindi restituisce il prodotto interamente, accertando e garantendo che sia conforme a tutti i requisiti del committente.

\subsubsection{Aspettative}
Il processo di validazione deve rispettare i seguenti punti:
\begin{itemize}
	\item identificazione degli obiettivi da validare;
	\item identificare ed applicare una strategia di validazione condivisa dal gruppo;
	\item valutare i risultati ottenuti dal processo di validazione.
\end{itemize}

\subsubsection{Attività}
Le attività da svolgere per validare il prodotto sono di esclusiva responsabilità del Responsabile di Progetto, e sono le seguenti:
\begin{itemize}
	\item accettare ed approvare il documento;
	\item rigettare il documento, fornendo indicazioni utili per la sua correzione.
\end{itemize}


