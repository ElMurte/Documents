Il \responsabile{} affiancato dagli \amministratori{} deve redigere il documento \PdPv{} in cui sono descritte le strategie utilizzate per affrontare le seguenti tematiche:

\begin{itemize}
    \item L'imposizione di vincoli temporali ben definiti \'{e} di fondamentale importanza per il successo del progetto, ci\'o si traduce nella necessit\'a di un'attenta pianificazione delle attivit\'a, che devono essere eseguite in diverse fasi progettuali. Per ogni fasi individuata, viene fornita una lista delle principali attivit\'a che il gruppo intende svolgere, la quale deve essere intesa solamente come traccia non esaustiva di tutte le attivit\'a che verranno svolte;
    \item Analisi dei potenziali rischi che possono manifestarsi nel corso dello sviluppo del progetto, categorizzati in base alla probabilit\'a che accadano e alla loro gravit\'a;
    \item Se non esplicitamente indicato nel contratto stipulato con il proponente, viene definito il modello di ciclo di vita appropriato al prodotto che si intende fornire, indicando le motivazioni che hanno portato alla scelta effettuata;
    \item Al fine di stimare i costi previsti, viene calcolato un preventivo, attraverso l'attribuzione di un costo ad ogni attivit\'a . Al termine di ogni fase progettuale vengono redatti dei consuntivi, con lo scopo di essere confrontati con il preveventivo precedentemente calcolato e attraverso questo confronto essere in grado di produrre preventivi sempre pi\'u accurati.
\end{itemize}

I \verificatori{} devono redigere il documento \PdQv{} in cui sono descritti i processi di verifica e le attivit\'a che vengono intraprese al fine di garantire la qualit\'a del materiale prodotto dal gruppo. In particolare il documento ha il compito di descrivere:

\begin{itemize}
    \item metriche per il controllo della qualit\'a del processo
    \item metriche per il controllo della qualit\'a del prodotto
    \item specifiche dei test
    \item resoconto delle attivit\'a di verifica
    \item valutazioni per il miglioramento
\end{itemize}