\documentclass{article}
\usepackage[utf8]{inputenc}

\title{Verbale meeting 31/10/2020}
\author{Alessio Trevisan }
\date{31 October 2020}

\begin{document}

\maketitle

\section{Scelta dei capitolati}
In seguito all'analisi e alla discussione di eventuali punti a favore e possibili problematiche associate ad ogni capitolato proposto, il team ha optato per la scelta del capitolato C1 come principale specifica di progetto sul quale operare, in quanto esso è affine agli interessi e alle ambizioni lavorative della maggioranza del team. Come possibili alternative a tale capitolato, sono stati selezionati i capitolati C6 e C3 (in ordine di preferenza).
\section{Disposizioni relative all'utilizzo dei software e di online tools per lo svolgimento del progetto }
Al termine del meeting il team ha deciso di utilizzare i seguenti programmi e online tools per l'ausilio alla realizzazione del progetto:\\
-Scheduling, suddivisione e assegnazione delle attività: Jira\\
-Versionamento del software e archiviod della documentazione: GitHub (l'utilizzo di eventuali programmi per il lavoro in locale è facoltativo, in linea di massima si consiglia l'utilizzo di Git e GitFlow)\\
-Riunioni in modalità telematica,raccolta di link utili: Discord
\end{document}

