\documentclass[a4paper, oneside, openany, dvipsnames, table]{article}
\usepackage[utf8]{inputenc}
\usepackage{../../../Shared/Sweleven}
\newcommand{\Titolo}{Studio di Fattibilit\`{a}}

\newcommand{\Approvatore}{TODO}
\newcommand{\Redattori}{\newline TODO \newline TODO \newline TODO
                        \newline TODO \newline TODO \newline TODO}
\newcommand{\Verificatori}{TODO}


\newcommand{\pathimg}{../../Shared/logo.png}

\newcommand{\Versionedoc}{0.0.5}

\newcommand{\Distribuzione}{Prof. Tullio Vardanega  \newline Prof. Riccardo Cardin  \newline Sweleven}

\newcommand{\Uso}{Interno}

\newcommand{\DescrizioneDoc}{TODO}

\newcommand{\NdPv}{Norme di Progetto \Versionedoc}
\newcommand{\Glossariov}{Glossario \Versionedoc}
\usepackage{comment}

% Ogni volta che si crea un documento partendo dal template nuovo, bisogn apportare le seguenti modifiche: 
%  Riga 13 comandi.tex per indicare i destinatari del documento
%  Riga 15 comandi.tex per indicare se il documento è interno o esterno 
%  Riga 21 comandi.tex per la descrizione del documento
%  Riga 11 comandi.tex per specificare la versione del documento.
%  Riga 11 comandi.tex per specificare la versione del documento.
%  Riga 3 introduzione.tex per lo scopo del documento
%  Riga 10 introduzione.tex riferimenti del documento.
% quindi cominciare la stesura del documento dei file presenti in sezioni/


\begin{document}
    \copertina{}

    {
    \rowcolors{2}{\evenRowColor}{\oddRowColor}
    \renewcommand{\arraystretch}{1.5}
    \centering
    \begin{longtable}{ c c  C{4cm}  c  c }
        \rowcolor{\primaryColor}
        \textcolor{\secondaryColor}{
        \textbf{Versione}}     & \textcolor{\secondaryColor}{\textbf{Data}}       & \textcolor{\secondaryColor}
        {\textbf{Descrizione}} & \textcolor{\secondaryColor}{\textbf{Nominativo}} & \textcolor{\secondaryColor}{\textbf{Ruolo}}                          \\

        % usare \verificatore per indicare verificatore
        % usare \redattore per indicare redatore
        % usare \responsabile per indicare responsabile

        %		L'ultimo evento deve essere sempre all'inizio della tabella
        0.0.1                  & 2020-12-01                                       & Stesura Iniziale del documento              & Elvis Murtezan & \responsabile{} \\
        X.X.X                  & XXXX-XX-XX                                       & Verifica                                    & TODO & \verificatore{} \\
        X.X.X                  & XXXX-XX-XX                                       & Descrizione edit                            & TODO & \redattore{}    \\
    \end{longtable}
}
    
    \newpage
    \tableofcontents
    
    \newpage
    \section{Informazioni generali}\label{sec:informazioni-generali}
    \section{Capitolato C2}
\subsection{Progettazione e sviluppo di una Real-time Gaming Platform(RGP)}
Il capitolato in questione si chiama "Progettazione e sviluppo di una \glo{Real-time Gaming Platform(RGP)}", il proponente è l'azienda Zero12 e i committenti sono \VT{} e \CR{}.

\subsection{Descrizione del capitolo}
L’azienda Zero12 propone la creazione di una RGP basata su un gioco a scorrimento multiplayer ed un'applicazione \glo{Android} o \glo{iOS} con finalità di \glo{testing} della RGP creata.
La finalità da parte del azienda è vedere alcune tecnologie di AWS in azione nell' ambito real-time ovvero la sincronizzazione dei giocatori in un gioco multiplayer a scorrimento verticale "infinito" e l'analisi in particolare di alcune tecnologie utilizzzate per la realizzazione della RGP.
\subsection{Prerequisiti e tecnologie coinvolte}
Prerequisiti:
\begin{itemize}
\item training sulle tecnologie coinvolte
\item analisi della tematica coinvolta RGP 
\end{itemize}
Tecnologie coinvolte:
\begin{itemize}
\item Utilizzo di \glo{Kotlin}/\glo{Swift} per lo sviluppo del applicativo mobile.
\item Utilizzo di un \glo{IDE} per la creazione di applicazioni mobile (Android o iOS);
\item \glo{Node.js} per lo sviluppo del server \glo{back-end};
\item \glo{Typescript}/\glo{Javascript} linguaggi di programmazione per il framework backend richiesto.
\item \glo{API} \glo{REST} attraverso le quali sia possibile interagire con l'applicativo;
\item AWS cloud services: 
\glo{AppSync}(è un servizio gestito che facilita lo sviluppo di API GraphQL) o \glo{Gamelift}(è un servizio gestito per server di gioco multiplayer basato su sessioni) 
\glo{DynamoDB}: database non relazionale, orientato a dati valore-chiave e documenti
\end{itemize}

\subsection{Vincoli}
\begin{itemize}
\item Creazione di un'applicazione Android o iOS di un gioco \glo{multiplayer}(da 2 a 6 giocatori) a scrolling verticale a piacere della tipologia \glo{PvE}"Player versus environment", con relativa interfaccia grafica.
\item Node.Js per le API (Typescript o Javascript) con cui comunicano i vari microservizi dell'applicazione
\item tecnologia AWS per la componente server: AppSync o Gamelift per la RGP e DynamoDB per la base di dati    
\item sincronizzazione \glo{UI} tra i giocatori (i giocatori vedono e giocano la stessa partita) ma vedono una sorta di fantasma del \glo{player-character} controllato dagli altri avversari.
\item consegna materiale pre Progettazione(design):analisi AppSync/Gamelift, Diagramma \glo{UML} \glo{use-cases},\glo{Schema Design} Basi di dati,documentazione API(dettagliata),piano di \glo{unit-test}
\item accesso al repository del codice sorgente dell'applicativo post produzione e \glo{Bug} \glo{Reporting}
\end{itemize}

\subsection{Aspetti positivi}
\begin{itemize}
\item Il prodotto richiesto risulta essere accattivante dal punto di vista didattico per la tematica e le tecnologie proposte.
\item Essendo Android molto diffuso la documentazione necessaria per realizzare l'applicazione è molto ricca e chiara;
\item L'azienda ha esposto in modo chiaro i vari vincoli ed i casi d'uso presenti nel capitolato.
\item L'azienda \`e disponibile a tenere delle sessioni di training sulle tecnologie AWS per spiegarne il funzionamento e l'utilizzo.
\end{itemize}
\subsection{Aspetti critici}
\begin{itemize}
\item Essendo cruciale la sincronizzazione tra giocatori, la RGP deve essere efficiente e la complessità che ne potrebbe derivare da questa potrebbe essere non banale.
\item ore di training nelle tecnologie non facilmente prevedibili
\end{itemize}
\subsection{Conclusioni}
La proposta del capitolato offerto dall'azienda Zero12 è stata accolta con grande interesse. Il gruppo è rimasto colpito e stimolato dalla possibilità 
di poter creare una RGP visto anche il crescente interesse nel settore a livello mondiale.Nonostante la tecnologia Android esista da molti anni è risultata particolarmente 
interessante da parte del gruppo, sia perché è supportata da un'ampia community di sviluppatori, sia perché per il gruppo è una tecnologia nuova che non viene 
trattata da nessun corso della laurea triennale. Dopo una prima analisi è stato in balottaggio per la scelta finale da parte del gruppo.

    
    \newpage
    \section{Verbale della riunione}\label{sec:verbale-della-riunione}
    \section{Capitolato C2}
\subsection{Progettazione e sviluppo di una Real-time Gaming Platform(RGP)}
Il capitolato in questione si chiama "Progettazione e sviluppo di una \glo{Real-time Gaming Platform(RGP)}", il proponente è l'azienda Zero12 e i committenti sono \VT{} e \CR{}.

\subsection{Descrizione del capitolo}
L’azienda Zero12 propone la creazione di una RGP basata su un gioco a scorrimento multiplayer ed un'applicazione \glo{Android} o \glo{iOS} con finalità di \glo{testing} della RGP creata.
La finalità da parte del azienda è vedere alcune tecnologie di AWS in azione nell' ambito real-time ovvero la sincronizzazione dei giocatori in un gioco multiplayer a scorrimento verticale "infinito" e l'analisi in particolare di alcune tecnologie utilizzzate per la realizzazione della RGP.
\subsection{Prerequisiti e tecnologie coinvolte}
Prerequisiti:
\begin{itemize}
\item training sulle tecnologie coinvolte
\item analisi della tematica coinvolta RGP 
\end{itemize}
Tecnologie coinvolte:
\begin{itemize}
\item Utilizzo di \glo{Kotlin}/\glo{Swift} per lo sviluppo del applicativo mobile.
\item Utilizzo di un \glo{IDE} per la creazione di applicazioni mobile (Android o iOS);
\item \glo{Node.js} per lo sviluppo del server \glo{back-end};
\item \glo{Typescript}/\glo{Javascript} linguaggi di programmazione per il framework backend richiesto.
\item \glo{API} \glo{REST} attraverso le quali sia possibile interagire con l'applicativo;
\item AWS cloud services: 
\glo{AppSync}(è un servizio gestito che facilita lo sviluppo di API GraphQL) o \glo{Gamelift}(è un servizio gestito per server di gioco multiplayer basato su sessioni) 
\glo{DynamoDB}: database non relazionale, orientato a dati valore-chiave e documenti
\end{itemize}

\subsection{Vincoli}
\begin{itemize}
\item Creazione di un'applicazione Android o iOS di un gioco \glo{multiplayer}(da 2 a 6 giocatori) a scrolling verticale a piacere della tipologia \glo{PvE}"Player versus environment", con relativa interfaccia grafica.
\item Node.Js per le API (Typescript o Javascript) con cui comunicano i vari microservizi dell'applicazione
\item tecnologia AWS per la componente server: AppSync o Gamelift per la RGP e DynamoDB per la base di dati    
\item sincronizzazione \glo{UI} tra i giocatori (i giocatori vedono e giocano la stessa partita) ma vedono una sorta di fantasma del \glo{player-character} controllato dagli altri avversari.
\item consegna materiale pre Progettazione(design):analisi AppSync/Gamelift, Diagramma \glo{UML} \glo{use-cases},\glo{Schema Design} Basi di dati,documentazione API(dettagliata),piano di \glo{unit-test}
\item accesso al repository del codice sorgente dell'applicativo post produzione e \glo{Bug} \glo{Reporting}
\end{itemize}

\subsection{Aspetti positivi}
\begin{itemize}
\item Il prodotto richiesto risulta essere accattivante dal punto di vista didattico per la tematica e le tecnologie proposte.
\item Essendo Android molto diffuso la documentazione necessaria per realizzare l'applicazione è molto ricca e chiara;
\item L'azienda ha esposto in modo chiaro i vari vincoli ed i casi d'uso presenti nel capitolato.
\item L'azienda \`e disponibile a tenere delle sessioni di training sulle tecnologie AWS per spiegarne il funzionamento e l'utilizzo.
\end{itemize}
\subsection{Aspetti critici}
\begin{itemize}
\item Essendo cruciale la sincronizzazione tra giocatori, la RGP deve essere efficiente e la complessità che ne potrebbe derivare da questa potrebbe essere non banale.
\item ore di training nelle tecnologie non facilmente prevedibili
\end{itemize}
\subsection{Conclusioni}
La proposta del capitolato offerto dall'azienda Zero12 è stata accolta con grande interesse. Il gruppo è rimasto colpito e stimolato dalla possibilità 
di poter creare una RGP visto anche il crescente interesse nel settore a livello mondiale.Nonostante la tecnologia Android esista da molti anni è risultata particolarmente 
interessante da parte del gruppo, sia perché è supportata da un'ampia community di sviluppatori, sia perché per il gruppo è una tecnologia nuova che non viene 
trattata da nessun corso della laurea triennale. Dopo una prima analisi è stato in balottaggio per la scelta finale da parte del gruppo.

    
    \newpage
    \section{Tracciamento delle decisioni}\label{sec:tracciamento-delle-decisioni}
    \section{Capitolato C2}
\subsection{Progettazione e sviluppo di una Real-time Gaming Platform(RGP)}
Il capitolato in questione si chiama "Progettazione e sviluppo di una \glo{Real-time Gaming Platform(RGP)}", il proponente è l'azienda Zero12 e i committenti sono \VT{} e \CR{}.

\subsection{Descrizione del capitolo}
L’azienda Zero12 propone la creazione di una RGP basata su un gioco a scorrimento multiplayer ed un'applicazione \glo{Android} o \glo{iOS} con finalità di \glo{testing} della RGP creata.
La finalità da parte del azienda è vedere alcune tecnologie di AWS in azione nell' ambito real-time ovvero la sincronizzazione dei giocatori in un gioco multiplayer a scorrimento verticale "infinito" e l'analisi in particolare di alcune tecnologie utilizzzate per la realizzazione della RGP.
\subsection{Prerequisiti e tecnologie coinvolte}
Prerequisiti:
\begin{itemize}
\item training sulle tecnologie coinvolte
\item analisi della tematica coinvolta RGP 
\end{itemize}
Tecnologie coinvolte:
\begin{itemize}
\item Utilizzo di \glo{Kotlin}/\glo{Swift} per lo sviluppo del applicativo mobile.
\item Utilizzo di un \glo{IDE} per la creazione di applicazioni mobile (Android o iOS);
\item \glo{Node.js} per lo sviluppo del server \glo{back-end};
\item \glo{Typescript}/\glo{Javascript} linguaggi di programmazione per il framework backend richiesto.
\item \glo{API} \glo{REST} attraverso le quali sia possibile interagire con l'applicativo;
\item AWS cloud services: 
\glo{AppSync}(è un servizio gestito che facilita lo sviluppo di API GraphQL) o \glo{Gamelift}(è un servizio gestito per server di gioco multiplayer basato su sessioni) 
\glo{DynamoDB}: database non relazionale, orientato a dati valore-chiave e documenti
\end{itemize}

\subsection{Vincoli}
\begin{itemize}
\item Creazione di un'applicazione Android o iOS di un gioco \glo{multiplayer}(da 2 a 6 giocatori) a scrolling verticale a piacere della tipologia \glo{PvE}"Player versus environment", con relativa interfaccia grafica.
\item Node.Js per le API (Typescript o Javascript) con cui comunicano i vari microservizi dell'applicazione
\item tecnologia AWS per la componente server: AppSync o Gamelift per la RGP e DynamoDB per la base di dati    
\item sincronizzazione \glo{UI} tra i giocatori (i giocatori vedono e giocano la stessa partita) ma vedono una sorta di fantasma del \glo{player-character} controllato dagli altri avversari.
\item consegna materiale pre Progettazione(design):analisi AppSync/Gamelift, Diagramma \glo{UML} \glo{use-cases},\glo{Schema Design} Basi di dati,documentazione API(dettagliata),piano di \glo{unit-test}
\item accesso al repository del codice sorgente dell'applicativo post produzione e \glo{Bug} \glo{Reporting}
\end{itemize}

\subsection{Aspetti positivi}
\begin{itemize}
\item Il prodotto richiesto risulta essere accattivante dal punto di vista didattico per la tematica e le tecnologie proposte.
\item Essendo Android molto diffuso la documentazione necessaria per realizzare l'applicazione è molto ricca e chiara;
\item L'azienda ha esposto in modo chiaro i vari vincoli ed i casi d'uso presenti nel capitolato.
\item L'azienda \`e disponibile a tenere delle sessioni di training sulle tecnologie AWS per spiegarne il funzionamento e l'utilizzo.
\end{itemize}
\subsection{Aspetti critici}
\begin{itemize}
\item Essendo cruciale la sincronizzazione tra giocatori, la RGP deve essere efficiente e la complessità che ne potrebbe derivare da questa potrebbe essere non banale.
\item ore di training nelle tecnologie non facilmente prevedibili
\end{itemize}
\subsection{Conclusioni}
La proposta del capitolato offerto dall'azienda Zero12 è stata accolta con grande interesse. Il gruppo è rimasto colpito e stimolato dalla possibilità 
di poter creare una RGP visto anche il crescente interesse nel settore a livello mondiale.Nonostante la tecnologia Android esista da molti anni è risultata particolarmente 
interessante da parte del gruppo, sia perché è supportata da un'ampia community di sviluppatori, sia perché per il gruppo è una tecnologia nuova che non viene 
trattata da nessun corso della laurea triennale. Dopo una prima analisi è stato in balottaggio per la scelta finale da parte del gruppo.

\end{document}