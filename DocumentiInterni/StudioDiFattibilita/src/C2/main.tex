\subsection{Informazioni generali}
\begin{itemize}
\item {\bf Nome:} 	EmporioLambda: piattaforma di e-commerce in stile Serverless;
\item {\bf Proponente:} Red Babel;
\item {\bf Committente:} Prof. Tullio Vardanega e Prof. Riccardo Cardin.
\end{itemize}


\subsection{Descrizione}
EmporioLambda mira alla creazione di una piattaforma web serverless di vendita e/o acquisto di beni o servizi che utilizzano internet e il trasferimento di denaro e dati che permettono queste transizioni.

\subsection{Finalit� del progetto}
Lo scopo del progetto � creare un'applicazione web di e-commerce dove i protagonisti saranno i clienti che acquisteranno diversi articoli sia fisici che di tipo servizi e i venditori dove metteranno a disposizione diversi beni. 

\subsection{Tecnologie interessate}
\begin{itemize}
\item {\bf Amazon Web Services(AWS):} insieme di servizi di cloud computing utili per il progetto tra cui:
\begin{enumerate}
\item {\bf Lambda:} AWS Lambda servizio di elaborazione serverless per l'esecuzione del proprio codice. I prezzi sono calcolati in base al tempo effettivo di elaborazione;
\item {\bf API gateway:} servizio API per la comunicazione con Lambda;
\item {\bf DynamoDB:} database non relazionale, orientato a dati valore-chiave e documenti;
\item {\bf CloudWatch:} servizio di monitoraggio e osservabilit� creato per ingegneri, sviluppatori, ingegneri responsabili dell'affidabilit� del sito (SRE) e manager IT DevOps;
\end{enumerate}
\item {\bf NodeJs:} piattaforma open source per l'esecuzione di codice JavaScript(TypeScript) server-side;
\item {\bf Next.js:} framework web di sviluppo front-end React open source che abilita funzionalit� come il rendering lato server e la generazione di siti web statici per applicazioni web basate su React. 
\end{itemize}
\subsection{Aspetti positivi}
\begin{itemize}
\item possibilit� di imparare nuovi linguaggi di programmazione come  Typescript;
\item possibilit� di imparare ad utilizzare servizi cloud come Amazon Web Services;

\end{itemize}

\subsection{Criticit� e fattori di rischio}
\begin{itemize}

\item L'azienda proponente ha sede all'estero, quindi la comunicazione con i referenti sar� meno agevole rispetto ai rapporti con un'azienda che ha sede nel territorio nazionale;
\item Il capitolato prevede l'utilizzo di tecnologie nuove, che quindi porteranno ad una mole di studio autonomo non indifferente.

\end{itemize}

\subsection{Conclusioni}
Nonostante tale capitolato abbia destato particolare interesse all'interno del team di lavoro, sia a livello tecnologico che di competenze curricolari, il gruppo si � mostrato pi� stimolato verso un altro progetto diversamente allettante.
