\subsection{Scopo del Documento}
Questo documento contiene la stesura dello studio di fattibilità riguardante i sette capitolati proposti, dove per ciascuno di essi vengono evidenziati i seguenti aspetti:
\begin{itemize}
    \item Titolo del capitolato;
    \item Descrizione generale;
    \item Prerequisiti e tecnologie coinvolte;
    \item Vincoli;
    \item Aspetti positivi;
    \item Aspetti critici.
\end{itemize}
Infine, per ogni capitolato vengono esposte le motivazioni e le ragioni per cui il gruppo ha scelto come progetto il capitolato C1 \NomeProgetto{}, in particolare l’interesse del gruppo a tale capitolato è stato cruciale nella decisone finale.
\subsection{Glossario}
Al fine di evitare ambiguità tra stakeholder, il gruppo Sweleven ha redatto un documento denominato \Glossariov, che definisce la terminologia utilizzata per la realizzazione del presente documento.
In tale documento, sono presenti tutti i termini tecnici, ambigui, specifici del progetto con le loro relative definizioni.
Un termine presente nel \Glossariov e utilizzato in questo documento viene indicato con un apice G alla fine della parola.
\subsection{Riferimenti}

\subsubsection{Normativi}
\begin{itemize}
\item \NdPv {}.
\end{itemize}

\subsubsection{Informativi}

\begin{itemize}
\item \textbf {Capitolato d'appalto C1 - Block Covid}\\
\url{https://www.math.unipd.it/~tullio/IS-1/2020/Progetto/C1.pdf}
\item \textbf {Capitolato d'appalto C2 - Emporio Lambda}\\
\url{https://www.math.unipd.it/~tullio/IS-1/2020/Progetto/C2.pdf}
\item \textbf {Capitolato d'appalto C3 - Gathering Detection Platform}\\
\url{https://www.math.unipd.it/~tullio/IS-1/2020/Progetto/C3.pdf}
\item \textbf {Capitolato d'appalto C4 - HD Viz }\\
\url{https://www.math.unipd.it/~tullio/IS-1/2020/Progetto/C4.pdf}
\item \textbf {Capitolato d'appalto C5 - PORTACS}\\
\url{https://www.math.unipd.it/~tullio/IS-1/2020/Progetto/C5.pdf}
\item \textbf {Capitolato d'appalto C6 - Realtime Gaming Platform}\\
\url{https://tinyurl.com/y6n4fj2j}
\item \textbf {Capitolato d'appalto C7 - Soluzioni di sincronizzazione Desktop}\\
\url{https://www.math.unipd.it/~tullio/IS-1/2020/Progetto/C7.pdf}

\end{itemize}
