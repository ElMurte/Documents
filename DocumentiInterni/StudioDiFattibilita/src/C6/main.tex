\section{Capitolato C2}
\subsection{Progettazione e sviluppo di una Realtime Gaming Platform(RGP)}
Il capitolato in questione si chiama "Progettazione e sviluppo di una \glo{Real-time Gaming Platform(RGP)}", il proponente è l'azienda Zero12 e i committenti sono \VT{} e \CR{}.

\subsection{Descrizione del capitolo}
L’azienda Zero12 propone la creazione di un \glo{Realtime system} per la \glo{sincronizzazione} di un gioco a scorrimento verticale,multiplayer ed un'applicazione \glo{Android} o \glo{iOS} con finalità di \glo{testing} del sistema software creato che gestisce le varie compnenti(RGP).
La finalità da parte dell’azienda è vedere alcune tecnologie di AWS in azione nell' ambito real-time ovvero la sincronizzazione dei giocatori in un gioco multiplayer a scorrimento verticale "infinito".
\subsection{Prerequisiti e tecnologie coinvolte}
Prerequisiti:
\begin{itemize}
\item training sulle tecnologie coinvolte
\item analisi della tematica coinvolta RGP 
\end{itemize}
Tecnologie coinvolte:
\begin{itemize}
\item Utilizzo di \glo{Kotlin}/\glo{Swift} per lo sviluppo del applicativo mobile.
\item Utilizzo di un \glo{IDE} per la creazione di applicazioni mobile (Android o iOS);
\item \glo{Node.js} per lo sviluppo del server \glo{backend};
\item \glo{Typescript}/\glo{Javascript} linguaggi di programmazione per il framework backend richiesto.
\item \glo{API} \glo{REST} attraverso le quali sia possibile interagire con l'applicativo;
\item AWS cloud services: 
\glo{AppSync}(è un servizio gestito che facilita lo sviluppo di API GraphQL) o \glo{Gamelift}(servizio gestito per server di gioco multiplayer basato su sessioni) 
\glo{DynamoDB}: database non relazionale, orientato a dati valore-chiave e documenti
\end{itemize}

\subsection{Vincoli}
\begin{itemize}
\item Creazione di un'applicazione Android o iOS di un gioco \glo{multiplayer}(da 2 a 6 giocatori) a scrolling verticale a piacere della tipologia \glo{PvE}"Player versus environment" infinito, con relativa interfaccia grafica.
\item Node.Js per le API (Typescript o Javascript) con cui comunicano i vari microservizi dell'applicazione
\item tecnologia AWS per la componente server: AppSync o Gamelift per la RGP e DynamoDB per la base di dati    
\item sincronizzazione \glo{UI} tra i giocatori (i giocatori vedono e giocano la stessa partita) ma vedono una sorta di fantasma del \glo{player-character} controllato dagli altri avversari.
\item consegna materiale: documentazione analisi AppSync/Gamelift pre \glo{Design}, Diagramma \glo{UML} \glo{use-cases},\glo{Schema Design} Basi di dati,documentazione API(dettagliata),piano di \glo{unit-test}
\item accesso al repository del codice sorgente dell'applicativo post produzione e \glo{Bug} \glo{Reporting}
\end{itemize}

\subsection{Aspetti positivi}
\begin{itemize}
\item Il prodotto richiesto risulta essere accattivante dal punto di vista didattico per la tematica e le tecnologie proposte.
\item Essendo Android molto diffuso la documentazione necessaria per realizzare l'applicazione è molto ricca e chiara;
\item L'azienda ha esposto in modo chiaro i vari vincoli ed i casi d'uso presenti nel capitolato.
\item L'azienda \`e disponibile a tenere delle sessioni di training sulle tecnologie AWS per spiegarne il funzionamento e l'utilizzo.
\item Dopo una prima analisi il servizio Gamelift dei vari servizi offerti da Amazon sembra facilitare parecchio lo sviluppo della RGP limitando il numero di microservizzi da sviluppare, partendo dal
presupposto che amazon essendo una azienda leader nel settore dei servizi cloud abbia creato questo servizio un motivo c'è e a nostro avviso è proprio 
ridurre la gestione lato server al minimo oltre alla crescente domanda in questo settore (come si può notare diverse aziende leader e startup del settore hanno già inizaito ad utilizzarlo).
Il vantaggio maggiore è che sposta il focus principale non nella RGP quasi completamente gestita da Gamelift che offre comunque le modifiche neccessarie, ma nello sviluppo del gioco multiplayer 
in questione, permettendo di ridurre costi e spostare le risorse intellettuali di un’azienda verso il gioco.
\end{itemize}
\subsection{Aspetti critici}
\begin{itemize}
\item Le ore di training del team nelle tecnologie richieste dal capitolato non sono facilmente prevedibili visto che alcuni di questi servizi sono abbastanza "nuovi" (come Gamelift 2016).
\item Lo sviluppo di un videogioco generalmente non è banale nel suo complesso.
\end{itemize}
\subsection{Conclusioni}
La proposta del capitolato offerto dall'azienda Zero12 è stata accolta con grande interesse. Il gruppo è rimasto colpito e stimolato dalla possibilità 
di poter creare una RGP visto anche il crescente interesse nel settore a livello mondiale. Nonostante la tecnologia Android esista da molti anni è risultata particolarmente 
interessante da parte del gruppo, sia perché è supportata da un'ampia community di sviluppatori, sia perché per il gruppo è una tecnologia nuova che non viene 
trattata da nessun corso della laurea triennale. Dopo una prima analisi è stato in balottaggio per la scelta finale da parte del gruppo.


