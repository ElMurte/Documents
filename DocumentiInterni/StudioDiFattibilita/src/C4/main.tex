
\subsection{Infomazioni generali}
Il capitolato in questione si chiama "HD Viz: visualizzazione di dati multidimensionali", il proponente è l'azienda Zucchetti ed i committenti sono il Prof. Tullio Vardanega ed il Prof. Riccardo Cardin.

\subsection{Descrizione del capitolo}
Il progetto richiede di creare una piattaforma web in grado di fornire diversi tipi di visualizzazioni grafiche di un gran numero di dati con molte dimensioni, a supporto della fase esplorativa di questi e della loro analisi. 
L’azienda proponente è particolarmente interessata a verificare la fattibilità di questo obbiettivo attraverso le tecnologie web.

\subsection{Prerequisiti e tecnologie coinvolte}
Prerequisiti:
\begin{itemize}
\item Studio delle librerie suggerite;
\item Organizzazione ed utilizzo di database;
\item Approfondimento delle principali tecnologie web.
\end{itemize}

Tecnologie coinvolte:
\begin{itemize}
\item \glo{D3.js}: libreria javascript per creare visualizzazioni dinamiche ed interattive partendo da dati organizzati;
\item \glo{Node.js} oppure \glo{Apache Tomcat}: sviluppo del server \glo{back-end};
\item \glo{Javascript}: linguaggio di scripting utilizzato per svolgere i diversi compiti ed eventi;
\item \glo{HTML}: linguaggio di markup per la realizzazione e strutturazione dei siti web;
\item \glo{CSS}: linguaggio per la formattazione e layout dei documenti web;
\item \glo{SQL} oppure \glo{NoSQL}: sviluppo e gestione di database;
\item \glo{Java}: linguaggio orientato agli oggetti per progettare applicativi;
\end{itemize}

\subsection{Vincoli}
\begin{itemize}
\item Uso delle tecnologie proposte al paragrafo precedente;
\item I dati da visualizzare devono avere almeno 15 dimensioni, deve però essere possibile rappresentare dati anche con un numero inferiore;
\item I dati devono poter essere inseriti sia tramite query ad un database, sia tramite file di tipo CSV;
\item Dovranno essere obbligatoriamente presenti almeno i seguenti tipi di visualizzazione:
\begin{itemize}
\item \glo{Scatter Plot Matrix}: fino ad un massimo di 5 dimensioni. Presentazione a riquadri disposti a matrice di tutte le combinazioni di grafici a dispersione, aiuta a trovare dimensioni con forti correlazioni;
\item \glo{Force Field}: grafico che traduce le distanze tra i dati in forze di attrazione e repulsione tra i punti nello spazio rappresentato, evidenziando i collegamenti e le strutture presenti;
\item \glo{Heat Map}: trasforma la distanza tra i punti in colori più o meno intensi. Deve consentire il riordinamento dei punti nel grafico, per evidenziare la struttura;
\item \glo{Proiezione Lineare Multi Asse}: posiziona i punti dello spazio multidimensionale in un piano cartesiano, riducendo a due dimensioni anche dati che ne hanno molte di più.
\end{itemize}
\end{itemize}
L’azienda proponente valuta inoltre positivamente i seguenti requisiti opzionali:
\begin{itemize}    
\item Implementazione di ulteriori tipi di visualizzazioni grafiche, adattate a dati di almeno 3 dimensioni;
\item Utilizzo di funzioni diverse da quelle previste di default da D3, per calcolare forze e distanze nelle relative visualizzazioni;
\item Analisi automatiche per evidenziare situazioni di particolare interesse;
\item Algoritmi di preparazione del dato precedenti alla sua rappresentazione grafica;
\item Eventuali altre proposte da parte del fornitore, considerate adeguatamente valide.
\end{itemize}

\subsection{Aspetti positivi}
\begin{itemize}
\item Buone competenze in \glo{data science} sono al giorno d’oggi molto utili e richieste in ambito professionale;
\item Arricchimento del bagaglio di conoscenze riguardo le tecnologie web e server;
\item Il proponente è piuttosto competente e professionale, ha inoltre esposto in modo chiaro i vari vincoli ed i casi d'uso presenti nel capitolato.
\end{itemize}
\subsection{Aspetti critici}
\begin{itemize}
\item Il processo di pulizia e classificazione dei dati potrebbe diventare piuttosto lungo e ripetitivo, quindi richiedere molto tempo al resto del lavoro;
\item Il progetto risulta non troppo stimolante;
\item Il capitolato, nel complesso, non ha riscosso particolare interesse da parte del gruppo.
\end{itemize}
\subsection{Conclusioni}
Seppur abbia degli indubbi aspetti positivi, il capitolato fin da subito non ha attirato l'interesse nella quasi totalità dei membri del team. Pertanto fin da subito non è stato preso particolarmente in considerazione.
